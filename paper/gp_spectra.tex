% !TEX TS-program = pdflatexmk
% mnras_template.tex
%
% LaTeX template for creating an MNRAS paper
%
% v3.0 released 14 May 2015
% (version numbers match those of mnras.cls)
%
% Copyright (C) Royal Astronomical Society 2015
% Authors:
% Keith T. Smith (Royal Astronomical Society)

% Change log
%
% v3.0 May 2015
%    Renamed to match the new package name
%    Version number matches mnras.cls
%    A few minor tweaks to wording
% v1.0 September 2013
%    Beta testing only - never publicly released
%    First version: a simple (ish) template for creating an MNRAS paper

%%%%%%%%%%%%%%%%%%%%%%%%%%%%%%%%%%%%%%%%%%%%%%%%%%

\documentclass[a4paper,fleqn,usenatbib]{mnras}

\usepackage{newtxtext,newtxmath}
\usepackage[T1]{fontenc}
\usepackage{ae,aecompl}
\usepackage{graphicx}	% Including figure files
\usepackage{amsmath}	% Advanced maths commands
\usepackage{amssymb}	% Extra maths symbols
\usepackage{bm}

\newcommand{\nb}{n_{\rm b}}
\newcommand{\nc}{n_{\rm c}}
\newcommand{\ns}{n_{\rm s}}
\newcommand{\prob}{{\rm P}}
\newcommand{\normal}{{\rm{N}}}
\newcommand{\uniform}{{\rm U}}
\newcommand{\dirichlet}{{\rm D}}
\newcommand{\invwish}{{\rm W}^{-1}}
\newcommand{\alphas}{{\bm a}}
\newcommand{\specmean}{{\bm m}}
\newcommand{\speccov}{{\bm S}}
\newcommand{\classprob}{{p}}
\newcommand{\classprobs}{{\bm p}}
\newcommand{\objspec}{{\bm s}}
\newcommand{\objclass}{{\kappa}}
\newcommand{\objclasses}{{\bm \kappa}}
\newcommand{\objdata}{\hat{\bm d}}
\newcommand{\objnoise}{{\bm N}}
\newcommand{\scalemat}{{\bm \Gamma}}
\newcommand{\wfmean}{{\bm w}}
\newcommand{\wfcov}{{\bm W}}
\newcommand{\condcov}{{\bm C}}

%Editing commands
\newcommand{\bdw}[1]{\textbf{\textcolor{magenta}{BDW: #1}}}
\newcommand{\smf}[1]{\textbf{\textcolor{blue}{SMF: #1}}}
\newcommand{\mkn}[1]{\textbf{\textcolor{red}{MKN: #1}}}

%%%%%%%%%%%%%%%%%%%%%%%%%%%%%%%%%%%%%%%%%%%%%%%%%%

\title[Gaussian Process Spectra]{Gaussian Process Spectra}
\author[S. M. Feeney et al.]{
Stephen M. Feeney,$^{1}$\thanks{E-mail: sfeeney@flatironinstitute.org}
Benjamin D. Wandelt,$^{1,2,3,4}$
and Melissa K. Ness$^{1,5}$
\\
$^{1}$Center for Computational Astrophysics, Flatiron Institute, 162 Fifth Avenue, New York, NY 10010, USA\\
$^{2}$Sorbonne Universit\'e, CNRS, UMR 7095,  Institut d'Astrophysique de Paris (IAP), 98 bis boulevard Arago, 75014 Paris, France\\
$^{3}$Sorbonne Universit\'e, Institut Lagrange de Paris (ILP), 98 bis boulevard Arago, 75014 Paris, France\\
$^{4}$Department of Physics and Astronomy, University of Illinois at Urbana-Champaign, 1002 W Green St, Urbana, IL 61801, USA\\
$^{5}$Department of Astronomy, Columbia University, Pupin Physics Laboratories, New York, NY 10027, USA
}

% These dates will be filled out by the publisher
\date{Accepted XXX. Received YYY; in original form ZZZ}

% Enter the current year, for the copyright statements etc.
\pubyear{2019}

% Don't change these lines
\begin{document}
\label{firstpage}
\pagerange{\pageref{firstpage}--\pageref{lastpage}}
\maketitle

% Abstract of the paper
\begin{abstract}
This is a simple template for authors to write new MNRAS papers.
The abstract should briefly describe the aims, methods, and main results of the paper.
It should be a single paragraph not more than 250 words (200 words for Letters).
No references should appear in the abstract.
Context: Little empirical characterisation of the information content in and limited approaches to data-driven modeling of stellar spectra, yet in an era where taking millions of spectra of stars. Therefore this affords tremendous opportunity to do these things.   
Aims: Make models of the spectra to understand how spectral element features are correlated, and how well we can predict missing parts of the spectra. Want to use many observations to make a model to provide de-noised and reconstructed flux inference for individual stars so as to make more precise measurements from individual spectra. 
Methods: Use Gaussian Process modeling
Results: Find that spectral absorption features of N elements are highly (but not perfectly) correlated by that all elements add additional information. Can use concert of measurements to predict missing very well
Conclusions :Use Gaussian process modeling to make first steps toward information content of element features. Furthermore we produce a generative model of the spectra and effectively de-noise to enable higher precision at lower SNR - important for optimising data and information extraction. 
\end{abstract}

% Select between one and six entries from the list of approved keywords.
% Don't make up new ones.
\begin{keywords}
keyword1 -- keyword2 -- keyword3
\end{keywords}

%%%%%%%%%%%%%%%%%%%%%%%%%%%%%%%%%%%%%%%%%%%%%%%%%%

\section{Introduction}

There is now a vast set of spectroscopic data and corresponding velocity, stellar parameter, individual abundance and age measurements, from surveys like APOGEE \citep{Majewski2017}, GALAH \citep{deSilva2015}, Gaia-ESO \citep{Gilmore2012}, RAVE \citep{Steinmetz2006}, SEGUE \citep{Yanny2009} and LAMOST \citep{Newberg2012}. An increasing number of large spectroscopic surveys will be starting observations in the coming years such as Sloan V \citep{Kollmeier2017}, WEAVE \citep{Bonifacio2016}, 4MOST \citep{deJong2019}, PFS \citep{PFS2018}, MOONS \citep{C2014} and Gaia RVS \citep{Gaia2016}. Large ($>$ 10$^5$ star) surveys like GALAH and APOGEE have enabled tremendous leaps forward already in our empirical characterization of the Milky Way and an improved understanding of its assembly. However, both of these surveys rely on expensive observations, integrating to high signal to noise (SNR) of up to SNR $>$ 100 per pixel. Such high SNR observations have been regarded as critical in the pursuit of precision abundances and to provide optimal chemical differentiation across the Galaxy. Such chemical differentiation may in principle allow the reconstruction of stars into their individual birth sites using abundances alone \citep[e.g.][]{BH2010}. Data-driven modeling approaches like The Cannon \citep{Ness2015} and also The Payne \citep{Ting2017} have shown that abundances can be measured at higher precision than previously, enabling equivalent precision to previous approaches with 1/4 to 1/9th of the observing time. This has in large part been the motivation for surveys like Sloan V's Galactic Genesys program using observations of stars at R=22,500, using the same instrument as APOGEE. Sloan V is obtaining an order of magnitude more stars within its Galactic Genesys  ($>$ 10$^6$) via more efficient sky sampling and by relaxing the SNR limit to $\sim$ 40 per pixel. Even such lower SNR spectra can yield  abundance precisions on the order of 0.05-0.1 dex given current approaches \citep{Ness2015, Ting2015}. Yet, what are the additional opportunities to use the ensemble of spectra to inform individual objects in the modeling of stellar spectra? Additionally, GAIA will deliver medium resolution spectra (R=11,000) for 7 million objects. Data-driven approaches have shown that an ensemble of individual abundances can also be derived at this resolution, and a factor of ten lower, given full spectral modeling \citep[e.g.][and Wheeler et al., in prep]{Casey2016, Ting2017}. Physically, this is well justified  \citep{Ting2018}, even for highly blended lines, as abundances can be measured from their impact on the entire spectral range as legitimately as from the individual element lines themselves. The coming decade will see increasing numbers of observations of stars by many orders of magnitude sets. We have the corresponding tremendous opportunity for data-driven modeling of these stars across our Galaxy. 

To date, there has been to date little work on the characterization and interpretation of the correlations within and dimensionality of the data \citep[see however][]{Ting2012, PJ2019, M2014}. Surveys currently might optimise their wavelength regions to attain an optimal number of element absorption features from which to make measurements, and aim for high signal to noise data to try to obtain high element abundance precision measurement. However, empirically the relationship between signal to noise, precision and the actual \textit{information gain} by observing any additional N+1 element given N measured elements, across a given wavelength region at a given resolution is unknown.  Furthermore, how is the information gain from measuring an increasing number of elements (presumably) variant across different elements and different stellar populations of field stars in the disk, halo and clusters? 

In this paper, we seek to determine if we can use many stars to generate a more precise representation of individual stars and in the process understand the information content of the data. We use stars observed by the APOGEE survey to build an extremely general and flexible empirical model from a large set of spectral data. We then examine how well this model reproduces individual spectra and interpret the correlation structure we measure in the data. We are able to examine and quantify the correlations in the spectral features on a per pixel level across different wavelength regions, which we select, which correspond to a set of abundance features. We explore and report the information content of these features. Specifically, we implement a Gaussian Process mixture model representation of the APOGEE data. This is a significant technical extension and expansion that builds upon existing progress in data-driven spectral modeling in the regime of large data sets. Our model of the data creates an effective de-noising of the individual spectra, that can predict masked or missing spectral regions. 

%We examine in detail the correlations between spectral pixels in element windows and the relative information gain attained by iterative abundance measurements. 
 
 

%Can we quantify the correlations between different wavelength regions in the data that correspond to different abundance absorption features? Furthermore, given most of the spectral data, can we predict some smaller wavelength region of the data using the quantification of these correlations from large numbers of stars?  

%We deliver a  spectra for each star. 







%Specifically, we examine empirically gained by adding an additional element of the same or different nucleosynthetic family in terms of additional spectral information? How correlated or anti-correlated are different spectral features in the data? With respect to the signal to noise required for precision abundance measurements, data-driven modeling like The Cannon and also The Payne have shown that the same precision can be obtained for APOGEE data with 1/4 to 1/9th of the observing time (1/2 to 1/3 of the SNR, so a SNR of around 40 compared to 100 for most abundances) and a precision near the theoretical limit (kramer-rao bound) can be obtained for APOGEE around a SNR of 40 (check), for most elements, which has motivated surveys like Sloan V to optimise their numbers of stars by collecting lower SNR data. 



%Is there any purpose of measuring all these elements? Empirically extra information in them? 

%-first look at the information content of the spectra
%-first characterisation and mapping of the element correlations

%state just red clump sample 

%[Ce, Cu, Mg, Ni, Mn, N, C, Ti, Si, Mg, Na, Fe, O, Ti, Ca, Nd, Fe, Si, Mn, Al, N, Ni, C]
%light: C, N
%light odd-Z: Na, Al
%alpha: Mg, Ti, Si, O
%iron-peak: Cu, Ni, Mn, Fe
%neutron capture r-process: Ce, Nd



%Fig 4 - trying to bring out the correlation structure in a slightly different way

%Fig 4 - perfectly uncorrelated would be random purpose and yellow because none of the pixels in the bin compare about any of the pixels in any other bin. If perfectly correlated yellow all up the top, purple all down the bottom. 

%-not parameterised in any physical way
%understanding of stellar spectra is truly rudimentary 
%this is part of a bigger effort in order to (under the assumption that everything is Guassian) allows us to learn the covariance at every point completely independent of any given physical template. 
%- question about the degrees of freedom (N*N)/2 degrees of freedom, diagonal of the covariance matrix
%- big thing is it shows the covariances
%-understand what the information content of the spectra%
%-inferring true spectra of every star.( N*N + Nclasses*2 + N*N)/2
%in the context of a much more sophistocated model 
%-runs into problems as matrix algebra (just gets really slow) 
%-new things
%-can do de-noising 

%Fig 6: information gain for each element and then pick the one that is most informative and then calculate the information gain on the window of interest given most informative element and each other one
%-log of the determinant of the covariance matrix of the window/ log of the covariance matrix of the window conditioned on other data
%-



%to do: For the Figure 6 - a list of the windows and the element families and then that is a whole little section . 

%Fig 2 - go down to just the window

%Fig 5: corner put SNR 

%state just red clump sample 

%How will we do this? By modeling the spectra as a data-driven Gaussian Process. What is a GP and what are relevant applications.

%%%%%%%%%%%%%%%%%%%%%%%%%%%%%%%%%%%%%%%%%%%%%%%%%%

In effectively pooling information about stars we achieve the following for the APOGEE spectra: \\
% reorder this 
%1. correlations
%2. iformation content
%3. masked regions
%4. make a metric to demonstrate - denoised spectra so can quantify in a section In terms of variances []
%ratio of standard deviations of the pink to the grey
\begin{enumerate}
\item Predict masked (unmeasured or contaminated) regions of the spectra to make, e.g., abundance measurements that would otherwise be impossible. This may be particularly valuable in APOGEE for neutron capture element measurements. For these elements in particular,  only one or two features exist from which to make measurements. Valuable elements like Nd and Ce add the additional dimensionality of the neutron-capture nucleosynthetic family (see Section 2). Some of these element features may also fall near the one of three chip gaps and may be missing in spectra and not others due to stellar velocities (citation. [Section 1]. Our modeling of the data can predict these regions when they are absent. 
\item Examine in detail the empirical correlations in the spectra, make quantitative measurements of these correlations and identify in practice which elements are positively and which are negatively affiliated (see Section 2). %[Section 2: might just want to add a plot of which are + and which are - correlated] 
\item De-noise to make precision measurements at lower SNR.  Generically, we can use the data-driven GP mixture model to de-noise and thus enhance the data, potentially enabling new discoveries. This should work for high SNR and low SNR regions; how useful this is depends on the size of the effects we may want to discover (see Section X). Our expectation is that this is particularly good for very weak features where our expectation is that our generated spectra can do a lot better with abundance measurements, similarly to that previously demonstrated with generative modeling (e.g. with The Cannon and The Payne).
\item Determine the most informative regions of spectra and make a measure of the information content of the data to, e.g., to determine whether we can retain sensitivity to abundances by observing a reduced spectral range and to understand if we have a set of N elements measured, do we gain information empirically by moving to N+1 measurements, for different element combinations, at the SNR of the APOGEE data and in general any spectral region and characteristics (e.g. resolution, SNR). 
\end{enumerate}


\section{Data} 

\textcolor{red}{mkn: sf  - I have random bits missing throughout including in intro with "X", please add where you see you know the answer.}

For our modeling we use the APOGEE red clump spectra from DR14 \citep{Majewski2017, Bovy2015}. These spectra comprise $29502$ stars with a mean SNR of 210 and range of SNR of 21-1775.  The contamination of red giant stars within this sample is on the order of 5-10 percent \citep{Bovy2015}. While our approach is generalisable across the entire stellar parameters, we to simplify our technical choices by restricting to the narrow temperature and gravity range of the red clump stars. These technical choices pertain to the inclusion and optimisation of multiple classes in the Gaussian Process modeling.

%These make an ideal test case for our modeling being restricted in their range of effective temperature and surface gravity so we can assume that our overall measurements are driven by abundance variations and not systematic variations across stellar parameters. The metallicity range of the red clump stars span --1.0 $<$ [Fe/H] $<$ 0.5 and they are distributed from R$_{GAL}$ = 4 to 20 kpc. 

The data have been downloaded from the APOGEE database and are radial velocity shifted back to the rest frame and continuum-normalised, with a slight SNR dependence on the continuum normalisation that we discover with our Gaussian Process modeling. The spectra cover the range 15100.80-16999.81 \AA, and comprise $\nb = 8575$ spectral bins. Repeated inversion of the $\nb \times \nb$ covariance matrices required for inference would be prohibitively slow, and we thus restrict our analysis to a set of spectral windows centred on lines confidently assigned to individual elements. These element windows have been chosen from the set of windows used to drive the APOGEE abundances in consultation with Jon Holtzman and Matthew Shetrone \citep{Holtzman2015, Shetrone2015}. Specifically, we process all spectral bins within $\pm 1.5$ \AA\ of the line centres specified in Table~\ref{tab:window_centres}, reducing the number of spectral bins to $\nb = 343$ and hence inversion time by a factor of $\sim15000$. %These spectral windows were selected using the APOGEE line list and communication with Jon Holtzman and Matthew Shetrone \citep{Holtz2015}. 

\begin{table}
    \centering
    \caption{The list of the 25 elements that we select for our spectral modeling and their corresponding central wavelength (in a vacuum) corresponding to Figure 4.}
    \label{tab:window_centres}
    \begin{tabular}{lcl}
        \hline
        element & window centre / \AA & elemental family \\
        \hline
        Al & 16723.500 & light odd-Z \\
        C & 15582.101 & light \\
        Ca & 16155.176 & alpha \\
        Ce & 15789.063 & neutron capture r-process \\
        Co & 16158.700 & iron peak \\
        Cr & 15684.264 & iron peak \\
        Cu & 16010.023 & iron peak \\
        Fe & 15495.100 & iron peak \\
        Ge & 16764.238 & neutron capture s-process \\
        K & 15167.081 & light odd-Z \\
        Mg & 15745.000 & alpha \\
        Mn & 15221.867 & iron peak \\
        N & 15321.871 & light \\
        Na & 16378.276 & light odd-Z \\
        Nd & 15372.342 & neutron capture r-process \\
        Ni & 15559.517 & iron peak \\
        O & 15760.300 & alpha \\
        P & 15715.930 & light odd-Z \\
        Rb & 15293.534 & neutron capture s-process \\
        S & 15482.319 & alpha \\
        Si & 15964.600 & alpha \\
        Ti & 15339.241 & alpha \\
        V & 15929.052 & iron peak \\
        Y & 15624.142 & neutron capture s-process \\
        Yb & 16502.973 & neutron capture s-process \\
        \hline
    \end{tabular}
\end{table}


\section{Methods}

Gaussian processes are a conceptually simple yet extremely powerful tool for regression and classification~\citep{Rasmussen_Williams}. Put briefly, a Gaussian process is a set of random variables whose joint distribution is multivariate normal, and is therefore fully specified by a mean function and covariance function. By their (Gaussian) nature, Gaussian processes permit simple, often analytically tractable, inference of their mean and covariance functions given (potentially noisy) observations, yielding flexible non-parametric fits to underlying trends in data and probabilistic predictions for new observations. As a result, Gaussian processes have found use throughout astronomy, from cosmology \citep{Bond_etal:1987} and cosmography \citep{Shafieloo_etal:2012} to models of instrumental systematics \citep{Gibson_etal:2012}, exoplanet populations \citep{DFM_etal:2014} and stellar spectra \citep{Czekala_etal:2017}.

In this work, we model the underlying ``true'' spectrum ($\objspec_i$), of the $i^{\rm th}$ APOGEE red clump star as a draw from a Gaussian process with a mean spectrum ($\specmean$) and covariance ($\speccov$) to be inferred from the data. In typical GP models, the covariance function is taken to be one of a number of standard kernels \citep{Rasmussen_Williams}, chosen to reflect known or assumed properties of the observation and/or physical process (e.g., stationarity, isotropy, or periodicity). In the following we do not assume an analytic form for our covariance function, but rather infer the correlations between the observed spectral bins, i.e., the individual elements of the covariance matrix. By doing so, we remove any potential for bias induced by a suboptimal kernel choice incorrectly enforcing stationarity, a single correlation length, or a particular line shape, for example. As a result, we can not make predictions for the spectra between the observed bins, though this would in principle be possible given stellar spectra observed on shifted or irregularly-sampled grids.

We assume the spectral data ($\objdata_i$) have been observed with Gaussian noise that is uncorrelated between spectral bins, yielding a diagonal noise covariance matrix ($\objnoise_i$) for each star. Masked pixels are assigned unit flux and (effectively) infinite noise uncertainties. To account for the fact that the red clump might consist of multiple distinct sub-populations (or one population whose distribution of true spectra is non-Gaussian), we allow for multiple classes to exist in our model, each described by its own Gaussian process. We assume noninformative priors on the variables defining these Gaussian processes, adopting an infinite uniform prior on each mean and a Jeffreys prior on each covariance matrix \citep[p73]{Gelman_etal:2013}. We infer the class membership of each star ($\objclass_i$) assuming they are drawn from categorical distributions with class probabilities ($\classprobs$) drawn, in turn, from a symmetric Dirichlet prior with concentration parameter $\alpha=1$. These priors state our beliefs that, {\it a priori}: no location is preferred for the mean spectra; no scale is preferred for the covariance between two spectral bins; and the stars are as likely to be spread evenly between classes as they are to be concentrated in a single class.

The data, model parameters, priors and likelihood fully specify our probabilistic model of the APOGEE red clump dataset. This model is naturally hierarchical, with some parameters describing populations and others individual stars. This hierarchical nature is made clear in Figure~\ref{fig:network_diagram}, in which we plot the model as a network diagram. In this diagram, random variables are shown as single black circles, observables as double black circles and fixed parameters as solid black dots. Links between parameters are indicated by arrows, with the probabilistic relationships defining the links contained within orange boxes. The direction of these arrows indicates the order in which parameters must be drawn in order to forward-model the data. Finally, populations of objects are contained within red rectangles or plates, with the indices denoting membership of the population defined in the bottom left of the plate.

\begin{figure}
	\includegraphics[width=\columnwidth]{bhm_plot.pdf}
    \caption{Network diagram for our hierarchical Bayesian model which is a graphical representation of our implemented modeling of the data. See Table 1 for the parameter descriptions.}
    \label{fig:network_diagram}
\end{figure}

For clarity, we set out our model's parameters, data and constants in Table~\ref{tab:params} and the probability distributions defining each link in the top section of Table.~\ref{tab:prob_dists}. The particular set of probability distributions chosen allow for the conditional distributions of each model parameter to be written analytically: these conditional distributions are specified in the bottom section of Table.~\ref{tab:prob_dists}. We are therefore able to use Gibbs sampling~\citep{Geman_and_Geman:1984} to estimate the joint posterior. Gibbs sampling is a special case of Metropolis-Hastings Monte Carlo~\citep{Hastings:1970} in which a single iteration consists of redrawing each parameter in turn from its conditional distribution based on the current state of the sampler. For example, in our case, we first update the class probabilities, then the class memberships, the true spectra, and finally each class's mean spectrum and covariance matrix. Drawing proposed updates from the conditional distributions means the acceptance probability is, by definition, unity, yielding a highly efficient sampler in even high-dimensional settings. The resulting sampler is written in Python and made publicly available on Github.\footnote{\href{https://github.com/sfeeney/ddspectra}{https://github.com/sfeeney/ddspectra}}

\begin{table}
    \centering
    \caption{Model parameters, data and constants.}
    \label{tab:params}
    \begin{tabular}{ll}
        \hline
        quantity & description \\
        \hline
        $\ns$ & number of stars (29502) \\
        $\nc$ & number of classes (default: 1) \\
        $\nb$ & number of spectral bins (default: 343) \\
        $\specmean_k$ & mean spectrum of $k^{\rm th}$ class \\
        $\speccov_k$ & intrinsic spectral covariance of $k^{\rm th}$ class \\
        $\classprob_k$ & $k^{\rm th}$ class probability: fraction of stars in $k^{\rm th}$ class \\
        $\alpha$ & concentration parameter of Dirichlet prior on \\
         & class fractions \\
        $\objspec_i$ & true spectrum of $i^{\rm th}$ star \\
        $\objclass_i$ & class assignment of $i^{\rm th}$ star \\
        $\objdata_i$ & observed spectrum of $i^{\rm th}$ star \\
        $\objnoise_i$ & noise covariance matrix of $i^{\rm th}$ star \\
        \hline
    \end{tabular}
\end{table}


\begin{table*}
    \centering
    \caption{Priors, likelihoods and conditional distributions for Gibbs sampling. In our simplified notation, $\uniform$, $\dirichlet$, $\normal$ and $\invwish$ denote uniform, Dirichlet, normal and inverse-Wishart distributions, respectively.}
    \label{tab:prob_dists}
    \begin{tabular}{lll}
        \hline
        distribution & form & process \\
        \hline
        $\prob\left(\specmean_k\right)$ & $\uniform\left(-\infty,\infty\right)$ & Prior on $k^{\rm th}$ class's mean spectrum \\
        $\prob\left(\speccov_k\right)$ & $\left| \speccov \right|^{-\left(\nb+1\right)/2}$ & Prior on $k^{\rm th}$ class's spectrum covariance \\
        $\prob\left(\classprobs|\alpha\right)$ & $\dirichlet\left(\alpha\right)$ & Prior on class probabilities \\
        $\prob\left(\objspec_i|\specmean,\speccov,\objclass_i\right)$ & $\normal\left(\specmean_{k=\objclass_i},\speccov_{k=\objclass_i}\right)$ & $i^{\rm th}$ object's spectrum as Gaussian Process \\
        $\prob\left(\objclass_i = k|\classprobs\right)$ & $\classprob_k$ & $i^{\rm th}$ object's class membership \\
        $\prob\left(\objdata_i|\objspec_i,\objnoise_i\right)$ & $\normal\left(\objspec_i,\objnoise_i\right)$ & Noisy, masked spectral measurements \\
        \hline
        $\prob\left(\specmean_k | \speccov_k, \objspec, \objclasses \right)$ & $\normal \left( \frac{1}{n_k} \sum_{\objclass_i = k} \objspec_i, \frac{1}{n_k} \speccov_k \right)$ & Conditional of $k^{\rm th}$ class's mean spectrum \\
%        $\prob\left(\speccov_k | \specmean_k, \objspec, \objclasses \right)$ & $\left| \speccov_k \right|^{-(n_k + \nb + 1)/2} \exp \left( -\frac{1}{2} {\rm tr} \left[ \scalemat_k \speccov_k^{-1} \right] \right)$, where & Conditional of $k^{\rm th}$ class's spectrum covariance \\
        $\prob\left(\speccov_k | \specmean_k, \objspec, \objclasses \right)$ & $\invwish \left(n_k, \scalemat_k \right)$, & Conditional of $k^{\rm th}$ class's spectrum covariance \\
         & where  $\scalemat_k = \sum_{\objclass_i = k} \left( \objspec_i - \specmean_k \right) \otimes \left( \objspec_i - \specmean_k \right)$ & \\
        $\prob\left(\classprob_k | \objclasses, \alpha\right)$ & $\dirichlet\left(\alphas\right)$, where $a_k = \alpha + n_k$ & Conditional of class probabilities \\
        $\prob \left( \objspec_i | \specmean_{k=\objclass_i}, \speccov_{k=\objclass_i}, \objdata_i, \objnoise_i \right)$ & $\normal \left( \wfmean_i, \wfcov_i \right)$, & Conditional of $i^{\rm th}$ object's spectrum \\
         & where $\wfcov_i = \left( \speccov_{k=\objclass_i}^{-1} + \objnoise_i^{-1} \right)^{-1}$ &  \\
         & and $\wfmean_i = \wfcov_i \left( \speccov_{k=\objclass_i}^{-1} \specmean_{k=\objclass_i} + \objnoise_i^{-1} \objdata_i \right)$ &  \\
        $\prob \left( \objclass_i = k | \specmean, \speccov, \classprobs \right)$ & $ \frac{ \exp \left( -\frac{1}{2} \left[ \chi^2_{i,k} + \ln \left| \speccov_k \right| \right] + \ln \classprob_k \right) }{ \sum_{k^\prime} \exp \left( -\frac{1}{2} \left[ \chi^2_{i,k^\prime} + \ln \left| \speccov_{k^\prime} \right| \right] + \ln \classprob_{k^\prime} \right) }$, & Conditional of $i^{\rm th}$ object's class membership \\
         & where $\chi^2_{i,k} = \left( \objspec_i - \specmean_k \right)^T \speccov_k^{-1} \left( \objspec_i - \specmean_k \right) $ &  \\
        \hline
    \end{tabular}
\end{table*}

%%%%%%%%%%%%%%%%%%%%%%%%%%%%%%%%%%%%%%%%%%%%%%%%%%

%%%%%%%%%%%%%%%%%%%%%%%%%%%%%%%%%%%%%%%%%%%%%%%%%%

\section{Results}

\subsection{Validation of Methodology: Predicting Unmeasured Spectral Regions}

To avoid the complications of comparing data gathered by different spectrographs, we validate our model and code by artificially masking a portion of one of our APOGEE spectra, namely the 15789 \AA\ Cerium (Ce II) window of our lowest-SNR star (2M18335753-1302240), with an SNR measurement of SNR = 21 as listed in the APOGEE allStar file. We chose this feature in particular as it is a high-value detection in the APOGEE spectral region, being an s-process element and only measured for a very small subset of the APOGEE stars to date \citep{Cunha2017}. This identification \citep{Cunha2017} and subsequent validation of the prospect of making measurements from this line has added the prospect of examining the s-process enrichment in the APOGEE survey, building on its reach and power in mapping the alpha, light and iron-peak elements across the disk and into the halo and local group \citep[e.g.,][]{Majewski2017, Nidever2014, Hayden2015, Weinberg2019}. Nine windows were identified in \citet{Cunha2017}: we select one (unblended) Ce II window here (the line centered on 15784.75 \AA\ in air, converted to the vacuum scale of the APOGEE spectra) for validation of our methodology. 


The measured data for this star in the artificially masked region are plotted in Figure~\ref{fig:recovery_test} as a solid black line. The 68\% credible interval for the posterior probability on the star's true spectrum is plotted as dark grey, with the corresponding prediction for the observed spectrum (which also takes into account the [known] uncertainty on the observations) plotted in light grey. This prediction (strictly speaking, the posterior predictive distribution of the measured data) is in excellent agreement with the measured data, indicating that our model is capable of in-painting masked regions without bias. Note, in addition, that the uncertainty on the true spectrum is much smaller than the measurement noise, demonstrating our method's ability to de-noise observed spectra (a phenomenon known as {\it shrinkage}).

This de-noising property is relevant in the regime of extracting information from both weak lines and lower signal-to-noise data than typically required. In addition to the neuron capture element, Ce, the APOGEE spectral region has been shown to contain a number of s-process Neodymium (Nd II) lines, which Hasselquist et al., 2016 estimates are detectable in $\approx$ 18 percent of APOGEE spectra using equivalent width fitting techniques. Our expectation is this fraction will greatly improve given our Gaussian process modeling of the spectral lines, which
%, similarly to our demonstration of the Ce line, can generate a spectral model with lower uncertainty than the measurement noise
can fit the true spectra of stars with lower uncertainty than the measurement noise. \mkn{SF: let's also point out the SNR of the star that we have Ce and Nd detections later on in the text.}

\begin{figure}
	\includegraphics[width=\columnwidth]{apogee_centers_final_29502_spc_rec_test_recovery_zoom.pdf}
    \caption{This Figure demonstrates the validation of our model and method via the recovery of an artificially masked portion of one star's spectrum: a 3 \AA\ region of spectrum centred on the Cerium line at 15789 \AA\ (see Table 1). We select a star with a SNR of 21 for this demonstration to highlight the performance of the model for what would traditionally be considered very low SNR data. The measured spectrum in this region is shown as a solid black line; once masked (dashed line) the flux is set to one, with infinite uncertainty. After fitting our model with the APOGEE dataset (including the remainder of this star's measured spectrum) we find that the true spectrum for this star should most likely fall in the dark grey region, and the measured spectrum (i.e., including instrumental noise) should fall in the light grey region. This is in excellent agreement with the data.}
    \label{fig:recovery_test}
\end{figure}

\subsection{APOGEE Inference: Feature Correlations Across the Abundance Windows}


Our inference produces samples of the probability, mean spectrum and covariance matrix for each class considered, and the true spectrum and class membership of each object. Focusing initially on the single-class case, we plot our covariance and mean inference in Figures~\ref{fig:inferred_cov} and~\ref{fig:gp_reals}, respectively. We plot the mean-posterior covariance matrix in Figure~\ref{fig:inferred_cov} (left panel). The covariance has strong off-diagonal structure, indicating that certain spectral features are highly correlated and anti-correlated. Its eigenspectrum also decays rapidly: only 241 of 557 eigenmodes have eigenvalues larger than 1/100 of the maximum. A low-rank approximation to the mean covariance retaining only these eigenmodes is plotted in the centre panel of Figure~\ref{fig:inferred_cov}, and the resulting residuals (multiplied by a factor of 500 to render visible) in the right panel. Exploiting this decaying eigenspectrum by assuming the covariance is rank deficient would greatly reduce computation time (by a factor of roughly 12 if 241 modes were retained). The most na\"ive method of reducing the covariance's rank---projecting the data onto the largest principal components of the sample covariance matrix prior to inference---unfortunately severely degrades performance. Modifying the hierarchical model (along the lines of \citet{Zhang_etal:2013}) to explicitly infer a rank-deficient covariance matrix is left to future work.\footnote{The structure of the covariance matrix also implies that certain kernels could potentially serve as useful covariance functions. Exploration of the utility of, for example, rational quadratic, Gibbs or mixtures of covariance functions~\citep{Rasmussen_Williams} is also left to future work.}

\begin{figure*}
	\includegraphics[width=2\columnwidth]{apogee_centers_final_29502_spc_win_wid_1p5_low_rank_covariance.pdf}
    \caption{Left: the mean-posterior covariance matrix ($\speccov$) of the 343 spectral pixels that we model, with the corresponding colour bar giving the magnitude of this covariance (in units of flux$^2$). The divergent colour map shows the most positive and negative covariances in red and blue respectively and zero covariance as white. This matrix demonstrates that the spectral pixels are highly correlated. Centre: a reduced-rank approximation of the mean-posterior covariance matrix, constructed using only those eigenvectors with eigenvalues within $10^{-4}$ of the largest. This represents a 30 percent reduction in the number of eigenvectors used to construct the mean-posterior covariance matrix. Right: the residual between the mean-posterior covariance matrix and its reduced-rank approximation, boosted by a factor of 500. The near-diagonal nature of the residuals indicates that the low-rank approximation cannot capture the additional (non-measurement) variance present in our mean-posterior covariance.}
    \label{fig:inferred_cov}
\end{figure*}
% to try to demonstrate how many 
% can speed things up in the inference by reducing the rank of the covariance matrix which is a 30 percent reduction in the number of pixels. 
%reduced rank puts null modes in the matrix - all structure in residuals is india
% the low rank version is missing variance on the diagonal 

\begin{figure*}
	\includegraphics[width=2\columnwidth]{apogee_centers_final_29502_spc_win_wid_1p5_gp_realizations.pdf}
    \caption{The mean-posterior mean spectrum ($\specmean$) of our Gaussian process model fit using the APOGEE data (black), along with 50 random realizations of potential true spectra ($\objspec$). These draws are coloured from purple to yellow according to their flux in the first spectral bin, and serve to demonstrate the correlations between pixels. Entirely uncorrelated data would show no structure in the colour gradient beyond the first bin; however, we see a clear stratification of yellow to purple as a function of the flux magnitude for most of the pixels.}
    \label{fig:gp_reals}
\end{figure*}

%using samples of covariance matrix and samples of sample posterior you can draw examples of the spectra itself 
%samples of mean and covariance and pick a randomly chosen sample and draw a realisation of that Gaussian process from that mean and covariance. 
%mean spectrum m (posterior) and some covariance s and then have samples of those; black ine is the sum over i of the m/i 
%realisations are gaussian with mean of mi and covafaince si) 
%m$_i$, s$_i$, \specmean = \frac{\sum{i=1}{n}}{m_i}{n}; r $\sim$ N(\specmean_i, \speccov_i) 
%\specmean\ and \speccov

The posterior mean of the mean spectrum is plotted in black in Figure~\ref{fig:gp_reals}. The mean spectrum is extremely well constrained: its 68\% credible interval is narrower than the width of the line. To illustrate the covariance structure captured by our model, we overlay 50 realizations drawn from our Gaussian process model conditioned on the APOGEE data, colour-coded by the value they take in the first spectral bin. These samples can be interpreted as examples of potential noiseless true spectra that could have led to the data. They illustrate the variability permitted by the model and highlight certain clear trends, most notably highly correlated differences in line depths.

We demonstrate our inference of the true spectra of individual stars in Figure~\ref{fig:inpainting_denoising_examples}, selecting six illustrative examples. From top to bottom, we pick out two spectra whose 15789 \AA\ Cerium windows are completely masked; two spectra whose 15372 \AA\ Neodymium windows are fully masked; and the two lowest signal-to-noise spectra. The APOGEE IDs for these stars are 2M00014650+7009328, 2M00031631+0042234, 2M04480027+3337594, 2M06053121-0631412, 2M18335753-1302240 and 2M18295507-0340512, with signal-to-noise ratios of SNR = 49, 63, 75, 41, 21 and 23, respectively. Each panel of Figure~\ref{fig:inpainting_denoising_examples} contains two shaded regions. The pink shaded area indicates the 1-$\sigma$ deviations from the measured spectra due to noise (these are infinitely wide when the spectrum is masked); the grey, the 68\% posterior credible intervals on the true spectra. Recall that we are inferring the true spectra at the measured spectral bins only. In this sense the smooth grey curves are perhaps misleading, as the posterior uncertainty is strictly infinite between data points.

Figure~\ref{fig:inpainting_denoising_examples} clearly demonstrates our ability to inpaint masked regions of spectra and denoise low signal-to-noise spectra. The inpainting results for the Cerium window are particularly encouraging. We are able to make precise (and very different) estimates of these two stars' spectra in the region of the Cerium line, permitting, in principle, measurements of their Cerium abundances where none was previously possible. The same is true for, for example, for the Aluminium lines of the third, fourth and fifth stars, along with the Oxygen and Germanium lines of the second, fifth and sixth stars. While we are also able to successfully inpaint the Neodymium windows for the third and fourth stars, our model infers very weak line profiles in both cases, making an abundance measurement challenging. Our ability to denoise the spectra is obvious for all stars considered: the uncertainties on the true spectra are in all cases smaller than the measurement uncertainty, permitting higher-precision abundance determination than previously possible. The Sodium line of the last two stars is a particularly good example of the potential for method to denoise spectra.

The results presented to this point assume that the APOGEE red clump stars belong to a single class (and their true spectra are therefore realizations of a single Gaussian process). We have experimented with allowing multiple classes, initializing the sampler with random class memberships; however, we find little impact on our final results. The sampler finds slight differences between the classes' mean spectra ($\specmean_k$) and covariances ($\speccov_k$), but these are driven by the initial randomized class memberships: very few stars leave one class for another during the sampling process, and those that do typically do so only once, in the sampler's first iteration. This is because the probability distribution used for drawing a star's class membership (Table~\ref{tab:prob_dists}, last row) drops exponentially with the squared distance between the star's true spectrum and each class's mean spectrum\footnote{Specifically, the Mahalanobis distance, or number of ``sigma'' the star's spectrum is from the class's mean.}. In very high dimensions, for almost all stars the distance to a new class is typically much larger than the distance to the current class, and the probability of transitioning to a new class is essentially zero. As such, we believe the class assignments are random and hence uninteresting. Exploring these high-dimensional clustering issues is left to future work. 

\begin{figure*}
	\includegraphics[width=2\columnwidth]{apogee_centers_final_29502_spc_win_wid_1p5_save_spectra.pdf}
    \caption{The measured and inferred spectra of six stars, all with low SNR (top to bottom: 49, 63, 75, 41, 21 and 23), selected to demonstrate our ability to both inpaint and denoise the data. The spectral regions shown are 3 \AA\ windows centred on the 25 elemental lines from Table~\ref{tab:window_centres}. The 68\% uncertainties on the observed spectra and inferred ``true'' spectra are shown as the pink and grey shaded regions, respectively (note the masked regions in the APOGEE spectra where the measurement uncertainties flare out to infinity). There is excellent agreement between the model and data. The first two spectra have completely masked Cerium lines (15789 \AA), and the inferred spectra clearly permit the measurement of Cerium abundances for these stars. The middle two stars' Neodymium (15372 \AA) lines are completely masked. Though the model again inpaints these regions successfully, the resulting line profiles are very weak, rendering recovery of this abundance unlikely. All other lines inferred by the model are denoised compared to the data, permitting the estimation of higher precision abundances.}
    \label{fig:inpainting_denoising_examples}
\end{figure*}

\subsection{The Measured Information Content in the Spectra}

We now turn to quantifying the information contained in each elemental window, with the aim of determining the regions of spectra most informative about particular elements of interest. We start with the mean-posterior covariance within each window, $\bar{\speccov}_{XX}$, as this describes the fundamental uncertainty with which we can predict the true spectrum of a new red clump star having observed our APOGEE sample. $X$ here denotes the spectral bins defining the elemental window of interest. We summarize this covariance matrix for six elemental windows ($X=\{{\rm C, Na, Mg, Fe, Ce, Ge}\}$: one from each elemental family) by plotting the root-mean-square (RMS) uncertainty,
\begin{equation}
\bm{\sigma}_X = \sqrt{{\rm diag} \left[\bar{\speccov}_{XX}\right]},
\end{equation}
in black in Figure~\ref{fig:single_element_errs}. For context, we overlay the typical measurement uncertainty\footnote{The square root of the average noise variance in each spectral bin, where the average is taken over stars in whose spectra the bin is not masked.} as a grey dashed line. This immediately demonstrates that our model of the APOGEE spectra allows us to make sub-noise predictions for some portions of a new star's spectrum without taking further data.


\begin{figure*}
	\includegraphics[width=\columnwidth]{apogee_centers_final_29502_spc_win_wid_1p5_c_conditional_stddevs.pdf}
	\includegraphics[width=\columnwidth]{apogee_centers_final_29502_spc_win_wid_1p5_na_conditional_stddevs.pdf}
	\includegraphics[width=\columnwidth]{apogee_centers_final_29502_spc_win_wid_1p5_mg_conditional_stddevs.pdf}
	\includegraphics[width=\columnwidth]{apogee_centers_final_29502_spc_win_wid_1p5_fe_conditional_stddevs.pdf}
	\includegraphics[width=\columnwidth]{apogee_centers_final_29502_spc_win_wid_1p5_ce_conditional_stddevs.pdf}
	\includegraphics[width=\columnwidth]{apogee_centers_final_29502_spc_win_wid_1p5_ge_conditional_stddevs.pdf}
    \caption{Root-mean-square uncertainties on the spectra within our illustrative set of elemental windows, centred on features due to C, Na, Mg, Fe, Ce and Ge, respectively. The black line shows the uncertainty on the predicted spectrum of a new APOGEE star having not observed any portion of its spectrum; the grey dashed line indicates the typical uncertainties due to APOGEE noise. The remaining lines show how the uncertainty decreases after having perfectly observed the $1 \le n \le 24$ most informative elemental windows of the new star's spectrum, coloured from purple (most informative) to yellow (least informative). The order in which elements are added is plotted in Figure~\ref{fig:single_element_information}. Note that the impact of adding observations decreases as the information gain curves of Figure~\ref{fig:single_element_information} become less steep.}
    \label{fig:single_element_errs}
\end{figure*}


To determine which windows are the most informative, imagine observing one window of this new star's spectrum (corresponding to, say, element $Y$) {\it without measurement error}. The long-range correlations present in the inferred covariance matrix (i.e., the fact that elemental abundances are determined by a finite number of physical processes) imply that by doing so we should better constrain the elemental window of interest. To quantify the information gained about element X by (perfectly) observing element Y, we calculate the conditional covariance matrix
\begin{equation}
\condcov_{XX|Y} = \bar{\speccov}_{XX} - \bar{\speccov}_{XY} \bar{\speccov}_{YY}^{-1} \bar{\speccov}_{YX}.
\end{equation}
The conditional covariance contains our full prediction for the uncertainty on window $X$ having observed window $Y$, but we must compress it in order to construct a useful metric for quantifying information gain. We do so by taking its determinant which, for covariance matrices, can be thought of as calculating the volume of the error ellipse on the quantities of interest. We define our information gain metric as
\begin{equation}
I = \log \frac{ \left| \condcov_{XX|Y} \right| }{ \left| \speccov_{XX} \right| } \le 0:
\end{equation}
the (logarithm of the) fractional reduction in uncertainty on the true spectrum in window $X$ obtained by observing window $Y$. Observing a new window can only add information, contracting the covariance matrix (or, in the worst-case scenario, leaving it unchanged), and thus $I$ cannot be greater than zero.

With this metric in hand, we can take each elemental window in turn and determine the information gained by observing each other window. The window $Y^1$ with the most negative $I$ is the most informative about our target window $X$; indeed, as our metric $I$ is symmetric, these two elements are the most informative about each other. We then repeat this process, conditioning on $Y^1$ {\it and} each other window in order to find the second most informative window, $Y^2$, continuing to add windows until we find the optimal order in which to build up information on the element of interest. We denote the list of the $n$ most informative elements $\bm{Y}^n = \{ Y^1, Y^2, \ldots, Y^n \}$; the covariance in window $X$ conditioned on these elements is $\condcov_{XX|\bm{Y}^n}$.

We plot the results of this process for the six illustrative elements in Figures~\ref{fig:single_element_errs} and~\ref{fig:single_element_information}. In Figure~\ref{fig:single_element_errs} we demonstrate how the RMS uncertainty within each elemental window shrinks as we condition on more and more information, now taking the RMS uncertainty to be
\begin{equation}
\bm{\sigma}_X = \sqrt{{\rm diag} \left[\condcov_{XX|\bm{Y}^n}\right]}.
\end{equation}
We plot the RMS uncertainties after conditioning on the $1 \le n \le \nb-1$ most informative windows as a series of solid curves, coloured from purple to yellow. We find that observing the most informative window, $Y^1$, significantly improves the uncertainty on the spectral window of interest. Conditioning on additional windows continues to add information, albeit with diminishing returns. After observing all other windows, the RMS uncertainty at the centre of the window of interest (i.e., directly over the elemental absorption line) has been reduced by at least a factor of two for all elements bar carbon; for germanium, this factor is closer to six.


\begin{figure*}
	\includegraphics[width=\columnwidth]{apogee_centers_final_29502_spc_win_wid_1p5_c_inf_gain.pdf}
	\includegraphics[width=\columnwidth]{apogee_centers_final_29502_spc_win_wid_1p5_na_inf_gain.pdf}
	\includegraphics[width=\columnwidth]{apogee_centers_final_29502_spc_win_wid_1p5_mg_inf_gain.pdf}
	\includegraphics[width=\columnwidth]{apogee_centers_final_29502_spc_win_wid_1p5_fe_inf_gain.pdf}
	\includegraphics[width=\columnwidth]{apogee_centers_final_29502_spc_win_wid_1p5_ce_inf_gain.pdf}
	\includegraphics[width=\columnwidth]{apogee_centers_final_29502_spc_win_wid_1p5_ge_inf_gain.pdf}
    \caption{These set of figures represent the information gains for the windows corresponding to the elements C, Na, Mg, Fe, Ce and Ge, as shown in the top of each sub-panel, given observations of the other 24 elemental windows. Each of the element windows is shown on the x-axis, ordered from the most informative window with respect to the primary selected element, that remains after conditioning on all previous element windows. To take the top right-hand sub-panel as an example: one would learn the most about the C window by observing Ni, then adding Ce, Ti {\it et cetera}. The y-axis quantifies this predictivity of successive elements as a measure of information gain, which is the change in entropy of the system provided by conditioning on another piece of information. Note the changing range in scale of the measure of information gain for the six different elements. The gain of the 24 additional element windows in the example of the the C primary window is less than the information gain for the Na primary windows. A larger overall difference is indicative of a larger overall information gain by measuring more elements. Note that the near exponential element-information gain relation, while becoming far less steep, does not entirely flatten, which indicates that each single element adds additional information to learn about the primary element. Although elements are highly correlated, there is a small intrinsic variance in each element that is not predicted in full by the remaining set of windows. }
    \label{fig:single_element_information}
\end{figure*}

% most predictive
% y axis quantifes the information gain. 
%the information gain is defined here as the difference between the log of the ratio of the determinant of the covariance matrix conditioned on observation in some other window and divided by the covariance matrix not condioned on any other obsservation. 
%log of the determinant of the covariance matirx tells  you the change in entropy of the system that you would get by conditioning on another piece of information.
% so it is very much an amoutn of information.
%low dimensional to some threshold
%two things - not looking at single elemental abundances because there might be other stuff in those windows, we are not using the same basis as abundances. Even if we were looking at abundances, not perfectly predictive. 
%intrinsic variance is coming out in this figure. 


Figure~\ref{fig:single_element_information} shows, for these exemplars, which element adds the most information. and then the next most informative, etc.. these tell us how the volume has shrunk, and therefore how the uncertainty on a particular portion of spectrum has shrunk. comments on elemental families.
What we find is essentially every single element adds information. Flat line would indicate no additional information, falls off exponentially at different rates for all the elements. Log scale so fractional gain for first N elements is XX and fractional gain for elements $>$ 5 is XX percent, relatively independent of which element is conditioned on. Elements coloured by family. Curve coloured in same way as Fig.~\ref{fig:single_element_errs}.

Importantly, we have the caveat that this is for elements measured from these particular spectral regions, could change as a function of some lines.

Figure~\ref{fig:all_element_information} we show full results: how much each window informs each other. as the information gain metric is symmetric the plotted matrices are too, but we retain the full matrix for prettiness. we present this information in two ways. in the first, we group elements by their families, sorting within families by increasing atomic number. two things to note. first is yellow peaks, which show particularly informative elemental pairs. the most informative pairs are ni-mn, fe-ti, mg-si, but overall the iron group elements predict each other well, as do the alpha elements (and, indeed, each other).

second thing to note is patterns of predictivity common to multiple elements: for example, the alpha-element and iron-group rows look similar. we make a first pass at sorting using this structure in the right hand panel. we quantify the similarity between two rows in the matrix using
\begin{equation}
d_{ij} = \sum_k |C_{ik} - C_{jk}|.
\end{equation}
using a Euclidean distance metric gives similar results. To sort the elements by similarity we use a simple greedy algorithm, approximating the global optimum through a series of locally optimal choices. We pick a first element and select the most similar element by finding the smallest $d$. We then take the second element as the comparator, calculating distances to all other remaining elements, and repeat until no elements remain. This approach is not guaranteed to find the global optimum, and indeed depends on the first element chosen. We therefore repeat the process with each element as the starting point and choose the first element whose sorted matrix minimizes the total distance between rows (define minimum distance, define sum over rows).

The results, shown in right hand panel, show similar conclusions to the family-grouped matrix. The iron-group elements are most similar, then the alpha elements. There is something of a clean break around rubidium and nitrogen beyond which the information gains are drop noticeably (though note that aluminium and copper are moderately informative about titanium, silicon, magnesium and cobalt: this may well be due to the sub-optimality of the greedy algorithm).

We repeat all of the above using broader 5-angstrom wide windows and present a version of Figure~\ref{fig:all_element_information} for these windows in Figure~\ref{fig:all_element_information_wide}. there are numerous things to note. first, the scale extends to larger negative values of information gain: these windows are broader, contain more information and are therefore more predictive. the optimal first element is now neodymium, not nickel, but the structure is still similar: the most informative elements are the iron-peak and alpha group, and these elements also have the most similar information gains and thus cluster in the plot. there is, however, more structure in the matrix (it's less blocky), thanks to the broader windows. this increases the chance of contamination: the yttrium entry being an excellent case in point. for these windows Y-Ni is the most informative pair: this is because the broad window around yttrium contains an iron line.

\begin{figure*}
	\includegraphics[width=\columnwidth]{apogee_centers_final_29502_spc_win_wid_1p5_sorted_inf_gains_fam_z.pdf}
	\includegraphics[width=\columnwidth]{apogee_centers_final_29502_spc_win_wid_1p5_sorted_inf_gains_abs_min_tot_dist.pdf}
    \caption{These two matricies show the information gains for pairs of elemental windows where a more negative measure represents a higher information gain. The information gain shown in the colourbar is defined as per  Figure~\ref{fig:single_element_information}. Therefore, the brighter (more yellow) the matrix element, the more informative the windows are about each other. At left, the elements are grouped according to their nucleosynthetic family, indicated by label colour of the elements listed on the x and y axes (see the text). \textcolor{blue}{mkn: add a description to the text of the different families and what color they appear in}. At right, the elements are grouped by their similar information gain. Note that the iron-peak family of elements shows the highest paired information gains confined within this family (Ni, Mn, Fe, Cr) - these elements are most predictive of one another, followed by the alpha-element family of elements (Ti, Si, Mg). However, the ordering of elements by their information gain similarity does not discretely separate elements into their nucleosynthetic families, particularly beyond the iron-peak and alpha-elements. {mkn: make sure to put in the text that this is windows and sensitive to whatever else is in there as not abudanance measurement, so need to be clear about being about windows and not a discrete measurement of abundance correlation etc.}. Information gains for all pairs of elemental windows: the brighter the matrix element, the more informative the windows are about each other. Left: elements grouped according to their family, indicated by label colour. Right: elements grouped by similar information gains.}
    \label{fig:all_element_information}
\end{figure*}

\begin{figure}
	\includegraphics[width=\columnwidth]{apogee_centers_final_29502_spc_sorted_inf_gains_abs_min_tot_dist.pdf}
    \caption{This Figure is the same as the right hand panel of Figure 8 (with elements grouped by similar information gains) except using wider 5 \AA windows, compared to the 3 \AA\ windows in Figure 8.  This Figure highlights the sensitivity of the information gain to the bandwidth selected to indicate each element in determination of which elements are most similar to one another. Although the similarity ranking has changed, the iron peak elements remain grouped as before and two of the alpha-elements (Mg and Si) are adjacent, as previously.\textcolor{blue}{mkn: make sure to highlight this aspect of our analysis in the text, that we show that we can denoise spectra for abundance measurements but here using the pixels themselves not isolated abundances to examine correlations and information, which is totally legitimate and useful just need to come back to this. }}
    \label{fig:all_element_information_wide}
\end{figure}

%through similarity

%%%%%%%%%%%%%%%%%%%%%%%%%%%%%%%%%%%%%%%%%%%%%%%%%%


\section{Discussion}


 In the context of the million star surveys coming online in the coming years, we seek to deliver  de-noised and ``betterized'' individual spectra for individual observations given the information we can learn from stellar ensembles. As such we aim to optimise the quality and quantity of measurements like chemical abundances and ages that can be made from this spectra. Currently limiting in this pursuit is the computational cost of the Gaussian process modeling, which here we run on a subset of APOGEE pixels. We select pixels from the full set of $\approx$ 8500 per star, which  cover 23 different element abundance absorption features. We demonstrate proof of concept and success of this approach on our subset of data.  Solving the technical challenges of feasibly scaling up to the full set of 8500 pixels for the APOGEE spectra will enable us to fully take advantage of this.
 
 Our examinination of the correlation between abundance windows in Section N indicates that there is value in measuring each individual abundance, \textcolor{red}{mkn: sf  - shows each have unique information but very marginal beyond some point - is it possible to state how marginal, say a factor of 100th or something like this by the time you have measured 5 elements, for pivoting on one element - do you see what I am getting at here? }

-use this de-noised data for physics-based measuring abundances (mkn)
-point to adding to Sloan V pipeline (mkn)
-technical aspects of the limited number of pixels being used. cubic scaling (sf)
- work to be done on multiple class version of this - modify sampler so that we get around this extremely low probability of changing class. (sf) 
-low-rank version discussion (sf) 


\section{Conclusions}
-proof of concept

-We demonstrate our methodology on the subset of 20,000 red clump stars in the APOGEE survey. Can imagine running across entire parameter space of APOGEE and providing generated model of set of abundance features for determining higher precision abudnances for and where data is missing, as well as weak (Ce, Nd) elements difficult to measure from spectra. Also expect a way to classify star formation histories using correlation matricies of elements. Expect for example that a Gaussian mixture model correlation matrix of globular cluster star would look very different to field star, element correlations are direct reflection of stellar phyiscs and star formation history that is dissimilar between different Milky Way populations. Could also be used as a classifier using spectral feature correlations themselves. 
-first characterisation / way of measuring correlation between elements/wavelength regions
-people could use all of the de-noised regions to measure all of the interesting elements in APOGEE e.g. Ce, Nd
Our work sets out avenues for continued exploration, and significant opportunity with respect to this is in the low resolution regime. Precision abundances can be derived from very low resolution spectra (Ho et al., 2016). We therefore must seek to better understand the information captured in the spectral features from an empirical point of view. 
-
- \textcolor{red}{publically available code online here} 


%%%%%%%%%%%%%%%%%%%%%%%%%%%%%%%%%%%%%%%%%%%%%%%%%%

\section*{Acknowledgements}

The Flatiron Institute is supported by the Simons Foundation.
Melissa Ness is supported in part by the sloan Foundation. We would like to thank Brice Menard (JHU) who was instrumental in bringing our team together to perform this work. 

%%%%%%%%%%%%%%%%%%%%%%%%%%%%%%%%%%%%%%%%%%%%%%%%%%

\bibliographystyle{mnras}
\bibliography{mknbib} % if your bibtex file is called example.bib

%%%%%%%%%%%%%%%%%%%%%%%%%%%%%%%%%%%%%%%%%%%%%%%%%%


% Don't change these lines
\bsp	% typesetting comment
\label{lastpage}
\end{document}

% End of mnras_template.tex